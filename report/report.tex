\documentclass[10pt,a4paper]{article}
\usepackage[utf8]{inputenc}
\usepackage{polski}
\usepackage{amsmath}
\usepackage{amsfonts}
\usepackage{amssymb}
\usepackage{graphicx}
\usepackage{verbatim}
\usepackage{minted}

\author{Gadawski Łukasz}
\title{Analiza statystyczna grafu przy użyciu standardowych narzędzi.}
\begin{document}
\maketitle

\section{Wykorzystane dane}
Numer mojego indeksu to 236655. W związku z tym reszta z dzielenia numeru przez 13 równa jest liczbie 3, a więc do mojej analizy przypisany został temat "Połączenia nerwowe nicienia \textit{Caenorhabditis elegans}".

\section{Operacje wstępne}
Wczytanie grafu w formacie \textit{GraphML} za pomocą pakietu \textit{networkx} odbywa się w prosty sposób. Wykonanie funkcji \textit{info()} wypisującej krótkie podsumowanie informacji o grafie daje następujący rezultat:

\begin{minted}{bash}
Name: C. Elegans neural network
Type: MultiDiGraph
Number of nodes: 297
Number of edges: 2359
Average in degree:   7.9428
Average out degree:   7.9428
Is directed:  True
\end{minted}

Typ \textit{MultiDiGraph} oznacza, że wczytany graf jest grafem skierowanym umożliwiającym przechowywanie duplikujących się krawędzi. Następnie podana jest liczba wierzchołków oraz krawędzi, a także średnie stopnie wejściowe oraz wyjściowe wierzchołków.

Aby usunąć zduplikowane krawędzie oraz przekształcić na graf nieskierowany należy stworzyć graf typu \textit{Graph} z wczytanego grafu. Po takiej operacji otrzymujemy następujące informacje na temat wczytanego grafu:

\begin{minted}{bash}
Name: C. Elegans neural network
Type: Graph
Number of nodes: 297
Number of edges: 2148
Average degree:  14.4646
Is directed:  False
\end{minted}

Co jest zgodne z naszymi oczekiwaniami. Zmniejszyła się nieznacznie liczba krawędzi co świadczy o tym, że w grafie wejściowym występowały zduplikowane krawędzie.

%\begin{figure}[H]
%	\centering
%	\includegraphics[scale=0.3]{vertex-element}
%	\caption{Analiza ze względu na ilość wierzchołków.	
%	\label{fig:vertex-element}}
%\end{figure}


\end{document}