\documentclass[10pt,a4paper]{article}
\usepackage[utf8]{inputenc}
\usepackage{polski}
\usepackage{amsmath}
\usepackage{amsfonts}
\usepackage{amssymb}
\usepackage{graphicx}
\usepackage{verbatim}
\usepackage{minted}

\author{Gadawski Łukasz}
\title{Analiza statystyczna grafu przy użyciu standardowych narzędzi.}

\begin{document}
\maketitle

\section{Wykorzystane dane}
Numer mojego indeksu to 236655. W związku z tym reszta z dzielenia numeru przez 13 równa jest liczbie 3, a więc do mojej analizy przypisany został temat "Połączenia nerwowe nicienia \textit{Caenorhabditis elegans}".

Jako dane wejściowe wykorzystano opis grafu w formacie \textit{GraphML}. W przypadku pakietu \textit{networkx} jest zdefiniowana gotowa metoda umożliwiająca wczytanie grafu w tym formacie. Do wykonania operacji w programie \textit{Pajek} w załączonym skrypcie stworzono konwerter grafu w tym formacie do pliku w formacie \textit{.net}, który może zostać wczytany przez program \textit{Pajek}.

\section{Networkx}
Wykonanie funkcji \textit{info()} na wczytanym grafie, powoduje wypisującej krótkiego podsumowania informacji o grafie:

\begin{minted}{bash}
Name: C. Elegans neural network
Type: MultiDiGraph
Number of nodes: 297
Number of edges: 2359
Average in degree:   7.9428
Average out degree:   7.9428
Is directed:  True
\end{minted}

Typ \textit{MultiDiGraph} oznacza, że wczytany graf jest grafem skierowanym umożliwiającym przechowywanie duplikujących się krawędzi. Następnie podana jest liczba wierzchołków oraz krawędzi, a także średnie stopnie wejściowe oraz wyjściowe wierzchołków.

Aby usunąć zduplikowane krawędzie oraz przekształcić na graf nieskierowany należy stworzyć graf typu \textit{Graph} z wczytanego grafu. Po takiej operacji otrzymujemy następujące informacje na temat wczytanego grafu:

\begin{minted}{bash}
Name: C. Elegans neural network
Type: Graph
Number of nodes: 297
Number of edges: 2148
Average degree:  14.4646
Is directed:  False
\end{minted}

Co jest zgodne z naszymi oczekiwaniami. Zmniejszyła się nieznacznie liczba krawędzi co świadczy o tym, że w grafie wejściowym występowały zduplikowane łuki, które zostały przekształcone na krawędzie.

Poprzez wykorzystanie odpowiedniej metody \textit{networkx.connected\_components(graph)} możliwe jest wyznaczenie składowych spójnych grafu, w tym przypadku występuje jedna składowa spójna. Rząd największej z nich wynosi 297 wierzchołków, natomiast jej rozmiar wynosi 2148 krawędzi. Stworzony w ramach projektu skrypt wypisuje te dane:

\begin{minted}{bash}
Number of connected components:  1
Range (nodes) of largest connected component:  297  nodes
Size (edges) of largest connected components:  2148  edges
\end{minted}

Następnie zaimplementowano wyznaczenie 5 wierzchołków o największej wartości wymaganych współczynników, uzyskując następujące wartości (każda para zawiera etykietę wierzchołka oraz wartość danego współczynnika):
\begin{itemize}
\item bliskości: \\
$[('n44', 0.30339720167657325), \\ ('n12', 0.05759006668189359), \\ ('n2', 0.04801549330577613),\\ ('n190', 0.044489436955661085),\\ ('n4', 0.031163449440266965)]$
\item pośrednictwa:\\
$[('n44', 0.597979797979798), \\('n12', 0.5220458553791887), \\('n2', 0.5165794066317626),\\ ('n86', 0.5085910652920962), \\('n3', 0.5008460236886633)]$

\item rangi:\\
$[('n44', 0.08673438527286116), \\('n2', 0.021427745128270354), \\('n12', 0.019912816778276513),\\ ('n172', 0.010832449312916758), \\('n86', 0.010682766687318981)]$
\end{itemize}

Dokonano wyznaczenie klik w rozpatrywanym grafie. Uzyskano następujące wartości, gdzie pierwsza wartość odpowiada rozmiarowi kliki, natomiast druga ilości klik danego rzędu:
\begin{itemize}
\item 0: 0
\item 1: 0
\item 2: 119
\item 3: 522
\item 4: 522
\item 5: 153
\item 6: 44
\item 7: 24
\item 8: 2
\end{itemize}

Zobrazowano ten rozkład również na poniższym wykresie:
\begin{figure}[H]
	\centering
	\includegraphics[scale=0.65]{kliki}
	\caption{Rozkład liczby klik w zależności od rzędu.
	\label{fig:kliki}}
\end{figure}

Na koniec dokonano grupowania aglomeracyjnego metodą \textit{comlete linkage} (średnica nowej grupy). Danymi wejściowymi do wykorzystanej metody jest macierz reprezentująca przekształcenie dla każdej pary wierzchołków wartości odległości pomiędzy każdą z par. Uzyskany dendrogram przedstawiono poniżej:
\begin{figure}[H]
	\centering
	\includegraphics[scale=0.65]{plot_dendrogram}
	\caption{Dendrogram.
	\label{fig:plot_dendrogram}}
\end{figure}

Arbitralnie podzieliłbym graf na linii odległości między wierzchołkami o wartości 60. Pozwoliłoby to na uzyskanie 7 grup.

\section{Pajek}
Wczytanie grafu w programu \textit{Pajek} było możliwe dzięki wykorzystaniu skryptu napisanym w języku python, który umożliwia zapis grafu do formatu \textbf{.net}. Uzyskano następujące podsumowanie:
\begin{minted}{bash}
Number of vertices (n): 297
----------------------------------------------------------
                                       Arcs          Edges
----------------------------------------------------------
Number of lines with value=1           1023              0
Number of lines with value#1           1336              0
----------------------------------------------------------
Total number of lines                  2359              0
----------------------------------------------------------
Number of loops                           0              0
Number of multiple lines                 14              0
----------------------------------------------------------

Density1 [loops allowed]    = 0.02674330
Density2 [no loops allowed] = 0.02683365

Average Degree = 15.88552189
\end{minted}

Wyniki są zbieżne z wynikami uzyskanymi przez wczytanie grafu przy użyciu pakietu \textit{networkx}, zarówno liczba łuków jak i średni stopień wierzchołka są z dokładnością do zaokrąglenie identyczne. W podsumowaniu zawarta została również informacja o zduplikowanych łukach, których występuje 14.

Usunięcie zduplikowanych łuków zostało zrealizowane poprzez transformację wczytanej sieci i usunięcie powielonych łuków. Operacje została przeprowadzona z ustawieniem dla pozostawionej krawędzi maksymalnej wartości, spośród łuków który były wielokrotne dla danej pary wierzchołków. Podstawowe informacje o grafie po tej operacji prezentują się następująco:
\begin{minted}{bash}
Number of vertices (n): 297
----------------------------------------------------------
                                       Arcs          Edges
----------------------------------------------------------
Number of lines with value=1           1013              0
Number of lines with value#1           1332              0
----------------------------------------------------------
Total number of lines                  2345              0
----------------------------------------------------------
Number of loops                           0              0
Number of multiple lines                  0              0
----------------------------------------------------------

Density1 [loops allowed]    = 0.02658459
Density2 [no loops allowed] = 0.02667440

Average Degree = 15.79124579
\end{minted}

Następnie dokanano transformację sieci i przekształcenie łuków na krawędzie, tym samym uzyskując graf nieskierowany. Po tej operacji uległa zmianie liczba krawędzi, a także średni stopień wierzchołka:
\begin{minted}{bash}
Number of vertices (n): 297
----------------------------------------------------------
                                       Arcs          Edges
----------------------------------------------------------
Number of lines with value=1              0            878
Number of lines with value#1              0           1270
----------------------------------------------------------
Total number of lines                     0           2148
----------------------------------------------------------
Number of loops                           0              0
Number of multiple lines                  0              0
----------------------------------------------------------

Density1 [loops allowed]    = 0.04870251
Density2 [no loops allowed] = 0.04886705

Average Degree = 14.46464646
\end{minted}

W dalszej kolejności została stworzona partycja, czyli grupa węzłów poprzez wyznaczenie składowych spójnych. W przypadku grafu nieskierowanego można wybrać zarówno opcję słabych jak i silnych składowych spójnych i tak uzyskany zostanie ten sam rezultat. W przypadku rozpatrywanego grafu istnieje jedna składowa spójna. Podsumowanie operacji przedstawiono poniżej:
\begin{minted}{bash}
==============================================================================
3. Strong Components of N3 [>=1] (297, comp.=1)
==============================================================================
Dimension: 297
The lowest value:  1
The highest value: 1

Frequency distribution of cluster values:

   Cluster      Freq     Freq%   CumFreq  CumFreq% Representative
 ----------------------------------------------------------------
         1       297  100.0000       297  100.0000 n38
 ----------------------------------------------------------------
       Sum       297  100.0000
\end{minted}

Na koniec dokonano wykreślenie grafu, który przedstawia się następująco:
\begin{figure}[H]
	\centering
	\includegraphics[scale=0.3, angle=90]{wykreslony-graf}
	\caption{Wykreślony graf.
	\label{fig:graf}}
\end{figure}

\section{Porówanie czasów}
Niestety dokładność pomiarów czasu wykonania operacji w przypadku programu \textit{Pajek} są na tyle niedokładne, że w przypadku każdej operacji uzyskano wartość 0:00:00. W przypadku zadań wykonanych przy użyciu pakietu networkx czasy przedstawiają się następująco:
\begin{itemize}
\item usunięcie zduplikowanych krawędzi oraz przekształcenie na graf nieskierowany: 0.007080078125
sek,
\item wyznaczenie składowych spójnych: 0.0146760940552 sek.
\end{itemize}

\section{Załączniki}
Do sprawzdania załączam skrypt w języku python umożliwiający uzyskanie przedstawionych rezultatów oraz dane na podstawie, których był wykonywany.

\end{document}
